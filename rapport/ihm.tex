%\documentclass[11pt,a4paper]{report}
%\usepackage[utf8]{inputenc}
%\usepackage{amsmath}
%\usepackage{amsfonts}
%\usepackage{amssymb}
%\usepackage{graphicx}
%\usepackage{geometry}
%\usepackage{float}
%\begin{document}

\section{Le menu principal}
Le menu principal va afficher tout les fonctions principales du programmes. Il s'agit de: \newline

\fbox{
\begin{minipage}{1\textwidth}
Projet C++: Réseaux Neurones \newline

====================== MENU =================== \newline
1. Mode Automatique\newline
2. Apprentissage\newline
3. Reconnaissance\newline
0. Quitter\newline
\end{minipage}}

\section{Mode automatique}
Si le choix est 1, le programme tourne avec es param\`etres par défaut. On a un ensemble d'image a et b qui est donné, et on va tester si on peut les différencier.

\begin{verbatim}
Faire tourner le programme avec les paramètres par défaut()
Apprentissage du jeu d'images A B contenant 30 images
15 premières images attendent A en réponse, les 15 dernières B
Apprentissage
0%
0.833333%
...
98.3333%
99.1667%
---------Echantillon d'apprentissage-----------
A a Image appr 0 :0.992383 Image appr 0 :-0.992422 
A a Image appr 1 :0.992364 Image appr 1 :-0.992361 
...
A a Image appr 22 :0.975503 Image appr 22 :-0.975471 
A a Image appr 23 :0.971372 Image appr 23 :-0.971625 
B b Image appr 24 :-0.989786 Image appr 24 :0.989907 
B b Image appr 25 :-0.980391 Image appr 25 :0.980473 
...
B b Image appr 47 :-0.990786 Image appr 47 :0.990857 
B b Image appr 48 :-0.993504 Image appr 48 :0.993523 
Ouverture du fichier de sauvegarde
Fermeture du fichier de sauvegarde
---------Echantillon de test-----------
B b ok ;) Image test 0 :-0.991597 Image test 0 :0.991683 
A a ok ;) Image test 1 :0.466849 Image test 1 :-0.467163 
A a ok ;) Image test 2 :0.119832 Image test 2 :-0.122299 
B b ok ;) Image test 3 :-0.988773 Image test 3 :0.988838 
A a ok ;) Image test 4 :0.986736 Image test 4 :-0.986759 
A a ok ;) Image test 5 :0.914367 Image test 5 :-0.914328 
A a ok ;) Image test 6 :0.0387813 Image test 6 :-0.0386743 
B b ok ;) Image test 7 :-0.986668 Image test 7 :0.986764 
B b ok ;) Image test 8 :-0.9753 Image test 8 :0.975194 
A a ok ;) Image test 9 :0.793187 Image test 9 :-0.794556 
A a ok ;) Image test 10 :0.791338 Image test 10 :-0.792993 
A a ok ;) Image test 11 :0.972948 Image test 11 :-0.973066 
A a ok ;) Image test 12 :0.992915 Image test 12 :-0.992907 
A a ok ;) Image test 13 :0.930262 Image test 13 :-0.930265 
B b ok ;) Image test 14 :-0.491248 Image test 14 :0.493489 
\end{verbatim}

%\fbox{
%\begin{minipage}{1\textwidth}
%Choix : 1\newline
%=============================================== \newline
%
%faire tourner le programme avec des parametres par défaut() \newline
%Apprentissage du jeu d'images A B contenant 30 images\newline
%15 premiére image attendent A en réponse, les 15 dérniéres B\newline
%Nombre de couches 4\newline
%---------Echantillon d'apprentissage-----------\newline
%A a Image appr 0 :0.992383 Image appr 0 :-0.992422\newline 
%A a Image appr 1 :0.992364 Image appr 1 :-0.992361 \newline
%A a Image appr 2 :0.99349 Image appr 2 :-0.993516 \newline
%A a Image appr 3 :0.992037 Image appr 3 :-0.992066 \newline
%A a Image appr 4 :0.990245 Image appr 4 :-0.990282 \newline
%A a Image appr 5 :0.988849 Image appr 5 :-0.98887 \newline
%A a Image appr 6 :0.98541 Image appr 6 :-0.985357 \newline
%A a Image appr 7 :0.990105 Image appr 7 :-0.990144 \newline
%A a Image appr 8 :0.984867 Image appr 8 :-0.984845 \newline
%A a Image appr 9 :0.98159 Image appr 9 :-0.98165 \newline
%A a Image appr 10 :0.974303 Image appr 10 :-0.974481\newline 
%A a Image appr 11 :0.971432 Image appr 11 :-0.971355\newline 
%A a Image appr 12 :0.976854 Image appr 12 :-0.976665\newline 
%A a Image appr 13 :0.972767 Image appr 13 :-0.972724\newline 
%A a Image appr 14 :0.979867 Image appr 14 :-0.980029\newline 
%A a Image appr 15 :0.97219 Image appr 15 :-0.971962 \newline
%A a Image appr 16 :0.989196 Image appr 16 :-0.989245\newline 
%A a Image appr 17 :0.991669 Image appr 17 :-0.991661\newline 
%A a Image appr 18 :0.982455 Image appr 18 :-0.98239 \newline
%A a Image appr 19 :0.989519 Image appr 19 :-0.98952 \newline
%A a Image appr 20 :0.980113 Image appr 20 :-0.980542\newline 
%A a Image appr 21 :0.992943 Image appr 21 :-0.992982\newline 
%A a Image appr 22 :0.975503 Image appr 22 :-0.975471\newline 
%A a Image appr 23 :0.971372 Image appr 23 :-0.971624\newline 
%B b Image appr 24 :-0.989786 Image appr 24 :0.989907\newline 
%B b Image appr 25 :-0.980391 Image appr 25 :0.980473\newline 
%B b Image appr 26 :-0.979724 Image appr 26 :0.979869\newline 
%B b Image appr 27 :-0.987759 Image appr 27 :0.987935\newline 
%B b Image appr 28 :-0.989009 Image appr 28 :0.989028\newline 
%B b Image appr 29 :-0.991693 Image appr 29 :0.991762\newline 
%B b Image appr 30 :-0.982201 Image appr 30 :0.982077\newline 
%B b Image appr 31 :-0.993486 Image appr 31 :0.993517\newline 
%B b Image appr 32 :-0.990494 Image appr 32 :0.990465\newline 
%B b Image appr 33 :-0.981015 Image appr 33 :0.980937\newline 
%B b Image appr 34 :-0.974582 Image appr 34 :0.974495\newline 
%B b Image appr 35 :-0.983483 Image appr 35 :0.983359\newline 
%B b Image appr 36 :-0.988139 Image appr 36 :0.988225\newline 
%B b Image appr 37 :-0.98496 Image appr 37 :0.985107 \newline
%B b Image appr 38 :-0.975069 Image appr 38 :0.975116\newline 
%B b Image appr 39 :-0.979455 Image appr 39 :0.979357\newline 
%B b Image appr 40 :-0.97496 Image appr 40 :0.975017 \newline
%B b Image appr 41 :-0.975214 Image appr 41 :0.975062\newline 
%B b Image appr 42 :-0.970069 Image appr 42 :0.970275\newline 
%B b Image appr 43 :-0.992843 Image appr 43 :0.992881\newline 
%B b Image appr 44 :-0.985963 Image appr 44 :0.986048\newline 
%B b Image appr 45 :-0.978272 Image appr 45 :0.978077\newline 
%B b Image appr 46 :-0.991868 Image appr 46 :0.991927\newline 
%B b Image appr 47 :-0.990786 Image appr 47 :0.990857\newline 
%B b Image appr 48 :-0.993504 Image appr 48 :0.993523\newline 
%\end{minipage}} \newline

\section{Apprentissage}
%\textbf{Choix 2 : Apprentissage}\newline
Si le choix est 2, le programme utilise le jeu d'apprentissage pour générer un nouveau réseau, et sauvegarde le réseau dans un fichier au choix.

\begin{verbatim}
Apprendre et sauvegarder le réseau
Entrer l'addresse de l'image
img/appr_a1.bmp
Entrer l'addresse souhaitée où l'on sauvegarde le réseau
res/test.txt
Apprentissage
0%
0.833333%
...
98.3333%
99.1667%
Ouverture du fichier de sauvegarde
Fermeture du fichier de sauvegarde
Votre réseau est sauvegardé à l'adresse res/test.txt
\end{verbatim}

%\fbox{
%\begin{minipage}{1\textwidth}
%======================= MENU ===================\newline
%
%1. Mode Automatique\newline
%2. Apprentissage\newline
%3. Reconnaissance\newline
%0. Quitter\newline
% 
%Choix : 2               \newline  
%================================================ \newline
%
%Apprendre et sauvegarder le reseau \newline
%Entrer l'addresse de l'image \newline
%img/appr\_a1.bmp \newline
% Entrer l'addresse souhaite ou on sauvegarde le reseau \newline
%res/test.txt \newline
%Apprentissage\newline
%Ouverture du fichier de sauvegarde  \newline
%Fermeture du fichier de sauvegarde \newline
% Votre reseau est sauvegarde a res/resultat.txt \newline
%\end{minipage}} \newline

\section{Reconnaissance}
Enfin, pour le choix numéro 3, le programme propose de reconnaitre un caractère.
\begin{verbatim}
Reconnaissance
Entrer l'addresse de l'image
img/appr_a1.bmp
Entrer un réseau
res/test.txt
Nombre de couches 4
C'est le caractère numéro :1 (a)
\end{verbatim}

%\fbox{
%\begin{minipage}{1\textwidth}
%======================= MENU ===================\newline
%
%1. Mode Automatique\newline
%2. Apprentissage\newline
%3. Reconnaissance\newline
%0. Quitter\newline
%  
%Choix : 3\newline
%=============================================== \newline
%
%Reconnaissance\newline
%Entrer l'addresse de l'image\newline
%img/appr\_a1.bmp\newline
%Entrer un reseau\newline
%res/test.txt\newline
%Nombre de couches 4\newline
%C'est le caractere numero :1 (a)\newline
%\end{minipage}}


%\end{document}